\newcommand{\productions}{P}
\newcommand{\nonterminals}{N}
\newcommand{\terminals}{\Sigma}
\newcommand{\startsymbol}{S}
\newcommand{\grammar}{(\nonterminals, \terminals, \productions, \startsymbol)}
\section{Parsing}
\label{Parsing}
Parsing is about analysing the structure of a sequence of tokens coming from some alphabet. We only give a brief overview of the problem here.
We begin by introducing some concepts of parsing in Section \ref{Parsing-Defs}. Then, in Section \ref{Parsing-Algebra}, we tie these concept together wwith the algebra from Section \ref{Section:Algebra}, and finally, in Section \ref{Parsing-TC}, we show that parsing is equivalent to computing the transitive closure of an upper triangular matrix.
In Section \ref{Section:Valiant}, we will then focus on a particular algorithm for computing the transitive closure, Valiant's algorithm, that we implement and prove the correctness of using Agda.

\subsection{Definitions}
\label{Parsing-Defs}
The goal of parsing is first to decide if a given sequence of tokens belongs to a given language, and second to describe its structure in the language.
For this, we consider the opposite process of generating a string in a given language. To do this, one uses a grammar for the language, which contains rules that can be used to build strings belonging to the language. %, or to try assign structural properties to a sequences of tokens.
\begin{Definition}
  A \emph{grammar} $G$ is a tuple $\grammar$, where 
  \begin{itemize}
  \item $\nonterminals$ is a finite set of nonterminals.
  \item $\terminals$, is a finite set of terminals, with $\nonterminals \cap \terminals = \emptyset$.
  \item $\productions$, is a finite set of production rules, written as $\alpha \to \beta$, where $\alpha$ and $\beta$ are sequences of terminals and nonterminals, and $\alpha$ contains at least one nonterminal.
  \item $\startsymbol \in \nonterminals$ is the start symbol.
  \end{itemize}
  We use upper case letters to denote nonterminals, lower case letters to denote terminals and Greek letters to denote sequences of both terminals and nonterminals.
\end{Definition}
A grammar generates a string of terminals by repeatedly applying production rules to the start symbol.
The language generated by a grammar is the set of strings of tokens it generates.

Parsing is then the process of taking a string and figuring out what (if any) sequence of expansions might have produced it. Often, one creates a datastructure annotating the string with the nonterminals generating the parts of the string.
\begin{Example}
  \label{Arithmetic}
  We present a simple grammar for a language of arithmetic expressions (which appears in slightly modified form in \cite{Lange-Leiss}) and give an example of string generation and one of parsing:
  \begin{itemize}
  \item $\terminals = \{1,\ldots , 9, +, *, (, )\}$,
  \item $\nonterminals = \{E, T, F, N\}$, for ``expression'', ``term'', ``factor'' and ``number'', respectively.
  \item The production rules are
    \begin{itemize}
    \item \label{p1} $E \to T$
    \item \label{p2} $E \to E + T$
    \item \label{p3} $T \to F$
    \item \label{p4} $T \to T * F$ 
    \item \label{p5} $F \to ( E )$
    \item \label{p6} $F \to N$
    \item \label{p7} $N \to i$, for $i = 1$, \ldots $9$. 
    \end{itemize}
  \item $S = E$.
  \end{itemize}
  We give an example of the generation of a string using this grammar in Table \ref{Str-Gen}: 
  \begin{table}
    \centering
    \begin{tabular}{l||l}
      Symbols & Explanation \\\hline
      $E$ & Start symbol
    \end{tabular}
    \caption{The generation of the string $...$ using the grammar in Example \ref{Arithmetic}\label{Str-Gen}}
  \end{table}
  
  To parse the above string, we apply the production rules backwards: 
  
\end{Example}
We are not going to consider arbitrary grammars in this report, so we give two restrictions to the definition above: 
\begin{Definition}
  A grammar is \emph{context free} if the left hand side of every production rule is a single nonterminal: $A \to \beta$.
\end{Definition}
\begin{Definition}
  A grammar is in (reduced) Chomsky Normal Form \cite{Chomsky} if the every production rule is of one of the following two forms:
  \begin{align*}
  A &\to a\\
  A &\to BC 
  \end{align*}
\end{Definition}
It is well known that any Context Free Grammar can be converted into a grammar in CNF (which generates the same language), with a size increase that is at most quadratic (the size of a grammar is the number of symbols contained \cite{Lange-Leiss}.
In the remainder of the report, we only consider grammars in Chomsky Normal Form.
\begin{Example}
  \label{CNF-Ex}
  We give a grammar in Chomsky Normal Form generating the same language as the one in Example \ref{Arithmetic}:
  \begin{itemize}
  \item \label{pc1} $E \to T$
  \item \label{pc2} $E \to E + T$
  \item \label{pc3} $T \to F$
  \item \label{pc4} $T \to T * F$ 
  \item \label{pc5} $F \to ( E )$
  \item \label{pc6} $F \to N$
  \item \label{pc7} $N \to i$, for $i = 1$, \ldots $9$. 
  \end{itemize}
\end{Example}
\subsection{Grammar as a nonassociative semiring}
\label{Parsing-Algebra}
The set of production rules for a grammar in Chomsky Normal Form which have the form $A \to BC$ could almost be used directly as a definition of multiplication among the nonterminals by replacing the arrows by equals signs:
\begin{equation*}
  BC = A
\end{equation*}
However, there are two problems with this. First, as we see in Example \ref{CNF-Ex}, there can be nonterminals $B$ and $C$ with no production $A \to BC$ (...). Second, there can be many different nonterminals that expand to the same thing: there can be $A_1$ and $A_2$ such that $A_1 \to BC$ and $A_2 \to BC$ are both productions.

The solution to these two problems is to instead consider \emph{sets} of nonterminals, with the following multiplication:
\begin{equation*}
  x \cdot y = \{A \st B \in x,\, C \in y,\, A \to BC \in P\}.
\end{equation*}

In general, this multiplication does not satisfy any algebraic axioms, it is not... \todo{THOMAS: refer back to example above -- once finished}.

Since we are considering sets, it is natural to choose set union as addition and hence $\emptyset$ as $0$.
\subsubsection{A specification for parsing}
In this section, we find a specification for the problem of parsing a string with a grammar in Chomsky Normal Form. Then, in the next section, we compare it to the specification used in \cite{Valiant}.
%In this section, we reformulate the parsing problem for a grammar in CNF form as the problem of finding the transitive closure of an upper triangle matrix.
%Valiant showed this when defining his algorithm in \cite{Valiant}, but he uses a different specification of the transitive closure, which is  less suitable for proofs in Agda.

One approach to parsing a string $w$ of length $n$ is to form a matrix $X$ containing the sets of all non-terminals that generate a substring: if we define $w[i,j)$ to be the substring starting at the $i$th symbol in $w$ and ending at the $j-1$st,we let: $X_{i j}$ be the set of all nonterminals generating of $w[i,j)$. The matrix formed this way is upper triangular since if $j \le i$, $X_{i j}$ is the set of all parses of an empty string.

Now, if we consider what nonterminals should be in the set $X_{i j}$, we note that:
\begin{itemize}
\item If $j = i + 1$, then $w[i,j)$ is a single token $a$. The only ways to generate $w[i,j)$ are using a production rules of the form $A \to a$, so $X_{i j}$ is the set of all $A$ such that $A \to a \in P$.
\item If $j > i + 1$, then $w[i,j)$ contains more than one token. The only ways to generate $w[i,j)$ are thus using a production rule $A \to BC$, where $B$ generates $w[i,k)$ and $C$ generates $w[k,j)$, for some $k$. For a fixed $k$, we find all nonterminals $A$ such that $A \to BC \in P$, where $B$ generates $w[i,k)$ and $C$ generates $w[k,j)$ by computing $X_{i k} \cdot X_{k j}$. Hence, 
            \begin{equation}
              X_{i j} = \bigcup_k X_{i k}\cdot X_{k j} = \sum_kX_{i k}X_{k j} = (XX)_{i j}
            \end{equation}
\end{itemize}
Combining the two points, we get:
\begin{equation}
  \label{TC}
  X = XX + C,
\end{equation}
where $C$ is the matrix whose only nonzero entries are $C_{i i+1} = \{A \st A \to w[i,i+1) \in P\}$. The above equation is the one we are going to use to prove the correctness of Valiant's algorithm in Section \ref{Valiant-proof}. 




\subsection{The specification used by Valiant}
In this section, we present the specification used in \cite{Valiant}, and prove that it is equivalent to our specification above \eqref{TC}. In particular, this implies that our specification defines a unique $X$, and proving that Valiant's algorithm computes it (in Section \ref{Valiant-proof}) proves that it exists.

In \cite{Valiant}, the following definition of the transitive closure of a matrix is used:
\begin{Definition}
  The (non-associative) \emph{transitive closure} of an upper triangular square matrix $C$ is the matrix $C^+$ defined by
  \begin{equation}\label{VSpec}
    C^+ = \sum_{i = 1}^\infty \nap{C}{i},
  \end{equation}
  where $\nap{C}{i}$ is the sum of all possible products containing $i$ copies of $C$, defined by:
  \begin{equation*}
    \nap{C}{1} = C \quad \text{and} \quad \nap{C}{i} = \sum_{j = 1}^{i - 1}\nap{C}{j}\nap{C}{i - j}.
  \end{equation*}
\end{Definition}
The sum in \eqref{VSpec} is actually finite, because as me mentioned in the end of Section \ref{Triangular-matrices}, any product of at least $n -1$ upper triangular matrices equals the zero matrix. Hence, if we show that $X$ defined in \eqref{TC} and $C^+$ defined in \eqref{VSpec} are equal, we can use \eqref{VSpec} to compute it, and it must be unique.
 \begin{Proposition}
   The two equations \eqref{TC} and \eqref{VSpec} are equivalent: $X$ satisfies 
   \begin{equation}
     X = XX + C
   \end{equation}
   if and only if 
   \begin{equation}
     X = \sum_{i = 1}^\infty \nap{C}{i}.
   \end{equation}
 \end{Proposition}
 \begin{proof}
   Assume first that $X = \sum_{i = 1}^\infty \nap{C}{i}= \sum_{i = 1}^{n-1}\nap{C}{i}$, where $n \times n$ is the size of $X$. It is clear that $X$ is upper triangular since $X$ is a sum of upper triangular matrices. And, 
   \begin{equation*}
     X X = 
     \left(\sum_{i = 1}^{n-1}\nap{C}{i}\right)\left(\sum_{j = 1}^{n-1}\nap{C}{j}\right) =
     \sum_{i = 1}^{n - 1}\sum_{j = 1}^{n-1}\nap{C}{i}\nap{C}{j} = 
     \sum_{i = 1}^{n - 1}\sum_{j = 1}^{n - i - 1}\nap{C}{i}\nap{C}{j},
   \end{equation*}
   where the last equality holds because the remaining terms contain more than $n-1$ factors, and hence equal zero.
   The above sum contains one copy of $\nap{C}{i}\nap{C}{j}$ for each $i = 1$, \ldots $n-1$ $j$ = $1$, $\ldots$, $n - i -1$. 
   Another way to sum these up (we can rearrange the sums since the addition is commutative) is to consider the sum $k = i + j$ and for each $k = 2$, \ldots, $n-1$ generate all products of $k$ factors. Hence, the above sum equals
   \begin{equation*}
     \sum_{k = 2}^{n-1}\sum_{l = 1}^{k-1}\nap{C}{l}\nap{C}{k-l} = \sum_{k = 2}^{n-1}\nap{C}{k},
   \end{equation*}
   and so, $XX + C = \sum_{k = 1}^{n-1}\nap{C}{k} = X$, and clearly.

   Next, we assume that $X$ is upper triangular and satisfies 
   \begin{equation*}
     \label{localdef}
     X = XX + C 
   \end{equation*}
   We define $R_k$ inductively by
   \begin{equation*}
     R_1 = XX + C \quad \text{ and } \quad R_{n + 1} = R_nR_n + C.
   \end{equation*}
   Since $X = R_1$, and if $X = R_n$, then by inserting $R_n$ in the right hand side of \eqref{localdef}, $X = R_nR_n + C = R_{n + 1}$, so by induction, $X = R_k$ for any $k$. We note also that after multiplying out (using the distributivity of multiplication over addition), $R_k$ is a sum of terms consisting of products of $X$ and $C$ (only). Now, we want to prove:
   \begin{enumerate}
   \item \label{1} For all $i$, $j \ge 1$, there is a $k$  such that $\nap{C}{i}\nap{C}{j}$ is a term in $R_k$.
   \item \label{2} If $\nap{C}{i}\nap{C}{j}$, for $i$, $j \ge 1$, is a term in $R_k$, then it is also a term in $R_{k + 1}$.
   \item \label{3} For every $k$, there are no structurally equal terms in $R_k$ (terms that are products of the same factors, in the same order).
   \item \label{4} The number of factors in terms containing $X$ in $R_k$ is at least $k + 1$. 
   \end{enumerate}
   From these facts, we can deduce that $X = \sum_{i = 1}^{n-1}\nap{C}{i}$: By \ref{1} and \ref{2}, there is a $R_k$ that contains all products $\nap{C}{i}\nap{C}{j}$ with $i < n - 1$, $j < n - 1$, and hence contains the sum $\sum_{i = 2}^{n-1}\nap{C}{i}$. By definition, it also contains $C$, so it contains $\sum_{i = 1}^{n-1}\nap{C}{i}$. Thus, (for example) $R_{k + n}$ also contains the whole sum. Since any term in $R_{k + n}$ involving $X$ contains at least $k + n + 1$ factors by \ref{4}, those terms are zero. So any other terms in $R_{k + n}$ are products of $C$s only. But any such term would either be a product of more than $n-1$  $C$s, and hence equal to zero, or a product of at most $n-1$ $C$s, and hence nonexistent since by \ref{3} there are no duplicates, and the term is in the sum $\sum_{i = 1}^{n-1}\nap{C}{i}$, so $X = R_{k + n} = \sum_{i = 1}^{n-1}\nap{C}{i}$.

We prove \ref{2} by induction on $i + j$. If $i+j = 2$, we get the statement that $CC$ is a term in every $R_k$, $k \ge 2$ (since it is a term in $R_2$) , but this is clearly true, since $R_{k - 1} = Y + C$, so that $R_{k} = (Y + C)(Y + C) + C = YY + YC + CY + CC + C$. If $i + j = n + 1$, and $\nap{C}{i}\nap{C}{j}$ is a term in $R_{k}$ for some $k \ge 2$, then from the definition of $R_k$, $\nap{C}{i}$ and $\nap{C}{j}$ are terms in $R_{k-1}$, and hence are terms in every $R_l$, $l \ge k-1$ by induction (since each of $\nap{C}{i}$ and $\nap{C}{j}$ are either equal to $C$ and hence a term in every $R_l$, or a sum of $\nap{C}{i'}{j'}$, with $i' + j' = i$ or $j$ which is at most $n$).

Next, we prove \ref{1}, again by induction on $i + j$. If $i + j = 2$, then $\nap{C}{i}\nap{C}{j} = CC$, which is a term in $R_2$. If $i + j = n + 1$, then there are $k_i$ and $k_j$ such that $\nap{C}{i}$ and $\nap{C}{j}$ are terms of $R_{k_i}$ and $R_{k_j}$ respectively (if $i$ or $j$ is $1$, then $R_1$ will do, otherwise we use induction, since $i$ and $j \le n$). Then if we let $k = \max(k_i,k_j)$, by \ref{2}, $R_k$ contains both, and hence, $R_{k+1}$ contains their product. 

We prove \ref{3} by induction on $k$. First, $R_1$ contains no structurally equal terms. Second, if $R_k$ contains no structurally equal terms, then if $R_{k+1}$ contains two structurally equal terms, they must both be products, say $X_1Y_1$ and $X_2Y_2$, and then, $R_{k}$ contains all four factors $X_1$, $Y_1$, $X_2$ and $Y_2$, but for $X_1Y_1$ to be structurally equal to $X_2Y_2$, their outermost parenthesis must be equal, and so we must have that $X_1$ is structurally equal to $X_2$ and $Y_1$ is structurally equal to $Y_2$, since otherwise the placement of the parenthesis would be different in the products. 

We prove \ref{4} by induction on $k$. In $R_1$, $XX$ is the only term containing $X$, and contains $2$ factors. If it is true for $k = n$, then when formin $R_{n+1}$, we multiply each term containing $X$, which contains at least $n + 1$ factors, by something and the result thus contains at least $n+2$ factors.
\end{proof}

Since we have proven that our specification \eqref{TC} is equivalent to Valiant's definition of transitive closure, \eqref{VSpec}, we will refer to our upper triangular matrix $X$ as the transitive closure of $C$. 

%%%% Comments
Although our specification \eqref{TC} is seemingly less ``computational'' than 
\eqref{VSpec}, which could be used to compute the transitive closure of $C$ by computing the value of the sum, ours is a lot simpler to use to derive Valiant's algorithm, as we do in Section \ref{Valiant-der}. In particular, using our specification together with a block matrix makes Valiant's algorithm fall out almost immediately, while simply proving the correctness of it using \eqref{VSpec} is a difficult task.

Additionally, we feel that our specification ties in with the problem of parsing a string much more naturally than \eqref{VSpec} does. By considering elements of the parse chart, it is clear that the chart should satisfy \eqref{TC}, while to go from parsing to \eqref{VSpec} involves using the fact that an element of the parse chart should contain all possible ``formally distinct'' sequences of nonterminals starting \cite{Valiant}.

%We might also mention that our specification avoids mentioning anything about the non-associativity of multiplication.


%To end this section, we note that other possible specifications of the transitive closure, similar to \eqref{JPTSpec}, that are equivalent to it if we assume associativity (and that addition is idempotent) fail to be correct without associativity, one such is:
%\begin{equation}
%  C^+ = C^+C + C,
%\end{equation}
%and adding the extra term $CC^+$, to get
%\begin{equation}
%  C^+ = C^+C + CC^+ + C
%\end{equation}
%fails to make the specification correct, since when expanding them, these two only ever produce bracketings of the form $(\cdots(CC)\cdots) C$ (and $C(\cdots(CC)\cdots)$.

