\section{Algebra}
We are going to introduce a bunch of algebraic things that will be useful either later or as point of reference. They will also be useful as an example of using agda as a proof assistant!

The first two sections are about algebraic structures that are probably already known. Both for reference, and as examples. Then we go on to more general algebraic structures, more common in Computer Science, since they satisfy fewer axioms (more axioms mean more interesting structure---probably---but at the same time, it's harder to satisfy all the axioms.
\subsection{Groups}
The first algebraic structure we will discuss is that of a group. We give first the mathematical definition, and then define it in Agda:
\begin{Definition}
A group is a set $G$ (sometimes called the \emph{carrier}) together with a binary operation $\cdot$ on $G$, satisfying the following:
\begin{itemize}
\item $\cdot$ is associative, that is, 
\item There is an element $e \in G$ such that $e \cdot g = g \cdot e = g$ for every $g \in G$. This element is the \emph{neutral element} of $G$.
\item For every $g \in G$, there is an element $g^{-1}$ such that $g \cdot g^{-1} = g^{-1} \cdot g = e$. This element $g^{-1}$ is the \emph{inverse} of $g$.
\end{itemize}
\end{Definition}
\begin{Remark}
One usually refers to a group $(G, \cdot, e)$ simply by the name of the set $G$.
\end{Remark}
\begin{Remark}
$G$ doesn't actually need to be a set. \todo{should this be noted}
\end{Remark}
An important reason to study groups that many common mathematical objects are groups. First there are groups where the set is a set of numbers:
\begin{Example}
  The integers $\Z$, the rational numbers $\Q$, the real numbers $\R$ and the complex numbers $\C$, all form groups when $\cdot$ is addition and $e$ is $0$.
\end{Example}
\begin{Example}
  The non-zero rational numbers $\Q\setminus{0}$, non-zero real numbers $\R\setminus{0}$, and non-zero complex numbers $\C\setminus{0}$, all form groups when $\cdot$ is multiplication and $e$ is $1$.
\end{Example}
Second, the symmetries of a 
% In Agda code, this is defined using a record:
%include /Code/Algebra/GroupDef.lagda

To prove that something is a group, one would thus
\subsection{Rings}
\subsubsection{Definition}
\subsubsection{Matrixes}
\subsection{Monoids}
\subsubsection{Definition}
\subsubsection{Cayley Table}
\subsection{Monoid-like structures}
\subsubsection{Definition}
\subsubsection{Cayley Table}
\subsection{Ring-like structures}
\subsubsection{Definition}
\subsubsection{Matrixes}
