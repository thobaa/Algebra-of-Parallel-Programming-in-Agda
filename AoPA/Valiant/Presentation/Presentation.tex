% -*- latex -*-
\documentclass[14pt, notes=hide]{beamer}

\usepackage{lmodern}


\usepackage{ucs}
%\DeclareUnicodeCharacter{2219}{\ensuremath{\bullet}} % ∙ 
\DeclareUnicodeCharacter{225C}{\ensuremath{\stackrel{\scriptscriptstyle {\triangle}}{=}}} % ≜
\DeclareUnicodeCharacter{2203}{\ensuremath{\exists}} % ∃
\DeclareUnicodeCharacter{2205}{\ensuremath{\varnothing}} % ∅
\DeclareUnicodeCharacter{2218}{\ensuremath{\circ}} % ∘
\DeclareUnicodeCharacter{00B7}{\ensuremath{\cdot}} % ·
\DeclareUnicodeCharacter{00B9}{\textsuperscript{l}} % ¹
\DeclareUnicodeCharacter{00D7}{\ensuremath{\times}} % × 
\DeclareUnicodeCharacter{00F7}{\ensuremath{\div}} % ÷
\DeclareUnicodeCharacter{0393}{\ensuremath{\Gamma}} % Γ
\DeclareUnicodeCharacter{0394}{\ensuremath{\Delta}} % Δ
\DeclareUnicodeCharacter{039E}{\ensuremath{\Xi}} % Ξ
\DeclareUnicodeCharacter{03A3}{\ensuremath{\Sigma}} % Σ
\DeclareUnicodeCharacter{03B1}{\ensuremath{\mathnormal{\alpha}}} % α
\DeclareUnicodeCharacter{03B2}{\ensuremath{\mathnormal{\beta}}} % β
\DeclareUnicodeCharacter{03B3}{\ensuremath{\mathnormal{\gamma}}} % γ
\DeclareUnicodeCharacter{03B4}{\ensuremath{\mathnormal{\delta}}} % δ
\DeclareUnicodeCharacter{03BB}{\ensuremath{\mathnormal{\lambda}}} % λ
\DeclareUnicodeCharacter{03BC}{\ensuremath{\mathnormal{\mu}}} % μ
\DeclareUnicodeCharacter{03BD}{\ensuremath{\mathnormal{\mu}}} % ν
\DeclareUnicodeCharacter{03BE}{\ensuremath{\mathnormal{\xi}}} % ξ
\DeclareUnicodeCharacter{03C0}{\ensuremath{\mathnormal{\pi}}} % π
\DeclareUnicodeCharacter{03C1}{\ensuremath{\mathnormal{\rho}}} % ρ
\DeclareUnicodeCharacter{03C3}{\ensuremath{\mathnormal{\sigma}}} % σ
\DeclareUnicodeCharacter{03C4}{\ensuremath{\mathnormal{\tau}}} % τ
\DeclareUnicodeCharacter{03C6}{\ensuremath{\mathnormal{\varphi}}} % φ
\DeclareUnicodeCharacter{03C7}{\ensuremath{\mathnormal{\chi}}} % χ
\DeclareUnicodeCharacter{03C8}{\ensuremath{\mathnormal{\psi}}} % ψ
\DeclareUnicodeCharacter{03C9}{\ensuremath{\mathnormal{\omega}}} % ω 
\DeclareUnicodeCharacter{10627}{\ensuremath{\lbana}} 
\DeclareUnicodeCharacter{10628}{\ensuremath{\rbana}} 
\DeclareUnicodeCharacter{2026}{\ensuremath{\ldots}}
\DeclareUnicodeCharacter{202F}{{\,}}
%\DeclareUnicodeCharacter{2115}{\ensuremath{\mathbbm{N}}} %not used (build in or smth)
\DeclareUnicodeCharacter{2191}{\ensuremath{\uparrow}} % ↑
\DeclareUnicodeCharacter{2192}{\ensuremath{\rightarrow}}
\DeclareUnicodeCharacter{21A6}{\ensuremath{\mapsto}}
\DeclareUnicodeCharacter{21D2}{\ensuremath{\Rightarrow}}
\DeclareUnicodeCharacter{2200}{\ensuremath{\forall}} % ∀
\DeclareUnicodeCharacter{2208}{\ensuremath{\in}} % ∈
\DeclareUnicodeCharacter{2209}{\ensuremath{\not\in}} % ∉
\DeclareUnicodeCharacter{220B}{\ensuremath{\ni}}
\DeclareUnicodeCharacter{2211}{\sum}% ∑
\DeclareUnicodeCharacter{2227}{\wedge}% ∧
\DeclareUnicodeCharacter{2228}{\vee}% ∨
\DeclareUnicodeCharacter{2245}{\ensuremath{\cong}} % ≅ 
\DeclareUnicodeCharacter{2248}{\ensuremath{\approx}} % ≈
\DeclareUnicodeCharacter{2260}{\neq}% ≠
\DeclareUnicodeCharacter{2261}{\equiv}% ≡
\DeclareUnicodeCharacter{2264}{\ensuremath{\le}} % ≤
\DeclareUnicodeCharacter{2265}{\ensuremath{\ge}} % ≥
\DeclareUnicodeCharacter{2282}{\ensuremath{\subset}} % ⊂
\DeclareUnicodeCharacter{2286}{\ensuremath{\subseteq}}
\DeclareUnicodeCharacter{2297}{\ensuremath{\otimes}} % ⊗
\DeclareUnicodeCharacter{22A2}{\ensuremath{\vdash}}
\DeclareUnicodeCharacter{22A4}{\ensuremath{\top}} % ⊤
\DeclareUnicodeCharacter{22A5}{\ensuremath{\bot}} % ⊥
\DeclareUnicodeCharacter{22A7}{\models} % ⊧ 
\DeclareUnicodeCharacter{22A8}{\models} % ⊨
\DeclareUnicodeCharacter{22A9}{\Vdash} % ⊩
\DeclareUnicodeCharacter{22C6}{\ensuremath{\star}}
\DeclareUnicodeCharacter{2308}{\ensuremath{\lceil}}
\DeclareUnicodeCharacter{2309}{\ensuremath{\rceil}}
\DeclareUnicodeCharacter{230A}{\ensuremath{\lfloor}}
\DeclareUnicodeCharacter{230B}{\ensuremath{\rfloor}}
\DeclareUnicodeCharacter{25A1}{\ensuremath{\square}} % □
\DeclareUnicodeCharacter{27E6}{\ensuremath{\llbracket}} % ⟦
\DeclareUnicodeCharacter{27E7}{\ensuremath{\rrbracket}} % ⟧
\DeclareUnicodeCharacter{27E8}{\ensuremath{\langle}} % ⟨
\DeclareUnicodeCharacter{27E9}{\ensuremath{\rangle}} % ⟩
\DeclareUnicodeCharacter{27F6}{{\longrightarrow}} % ⟶
\DeclareUnicodeCharacter{27F7}{{\longleftrightarrow}} % ⟷
\DeclareUnicodeCharacter{8499}{\mathcal{M}}
\DeclareUnicodeCharacter{8592}{\ensuremath{\leftarrow}}
\DeclareUnicodeCharacter{8594}{\ensuremath{\rightarrow}}
\DeclareUnicodeCharacter{8614}{\ensuremath{\leadsto}} % ↝
\DeclareUnicodeCharacter{8644}{\ensuremath{\leftrightarrows}} % ⇆
%\DeclareUnicodeCharacter{8718}{\ensuremath{\blacksquare}}use --agda for lhs2Tex
\DeclareUnicodeCharacter{8746}{\ensuremath{\cup}}
%\DeclareUnicodeCharacter{8759}{::} % ∷ use --agda for lhs2Tex
\DeclareUnicodeCharacter{8797}{\mathrel{\mathop:}=}
\DeclareUnicodeCharacter{8799}{\ensuremath{\stackrel{\scriptscriptstyle ?}{=}}} % ≟
\DeclareUnicodeCharacter{8873}{\Vdash}
\DeclareUnicodeCharacter{8988}{\ensuremath{\lceil}}
\DeclareUnicodeCharacter{8989}{\ensuremath{\rceil}}
\DeclareUnicodeCharacter{9657}{\ensuremath{\triangleright}}
\DeclareUnicodeCharacter{9667}{\triangleright{}}
\DeclareUnicodeCharacter{9669}{\ensuremath{\triangleleft}}
\DeclareUnicodeCharacter{9725}{\ensuremath{\square}}
\DeclareUnicodeCharacter{9733}{\ensuremath{\star}}   % ★

% thomas (don't know if things change when using latex + agda or not)
\DeclareUnicodeCharacter{8469}{\ensuremath{\mathbb{N}}}
\DeclareUnicodeCharacter{8252}{\ensuremath{!!}} % ‼
\DeclareUnicodeCharacter{955}{\ensuremath{\lambda}}
\DeclareUnicodeCharacter{931}{\ensuremath{\Sigma}}
\DeclareUnicodeCharacter{739}{\ensuremath{^\times}}


\newcommand{\trc}[1]{#1^{\times}}
\DeclareMathOperator{\overlap}{Overlap}
\DeclareMathOperator{\tc}{TC}
% format ->         = "\to "
% format <-         = "\leftarrow "
% format =>         = "\Rightarrow "
% format \          = "\lambda "
% format !!         = "\mathbin{!!}"
% format `quot`     = "\mathbin{\Varid{`quot`}}"
% format `rem`      = "\mathbin{\Varid{`rem`}}"
% format `div`      = "\mathbin{\Varid{`div`}}"
% format `mod`      = "\mathbin{\Varid{`mod`}}"
% format ==         = "\equiv "
%% ODER: format ==         = "\mathrel{==}"
% format /=         = "\not\equiv "
%% ODER: format /=         = "\neq "
% format <=         = "\leq "
% format >=         = "\geq "
% format `elem`     = "\in "
% format `notElem`  = "\notin "
% format &&         = "\mathrel{\wedge}"
% format ||         = "\mathrel{\vee}"
% format >>         = "\sequ "
% format >>=        = "\bind "
% format =<<        = "\rbind "
% format undefined  = "\bot "
% format not	   = "\neg "
%
%
\makeatletter
\@ifundefined{lhs2tex.lhs2tex.sty.read}%
  {\@namedef{lhs2tex.lhs2tex.sty.read}{}%
   \newcommand\SkipToFmtEnd{}%
   \newcommand\EndFmtInput{}%
   \long\def\SkipToFmtEnd#1\EndFmtInput{}%
  }\SkipToFmtEnd

\newcommand\ReadOnlyOnce[1]{\@ifundefined{#1}{\@namedef{#1}{}}\SkipToFmtEnd}
\usepackage{amstext}
\usepackage{amssymb}
\usepackage{stmaryrd}
\DeclareFontFamily{OT1}{cmtex}{}
\DeclareFontShape{OT1}{cmtex}{m}{n}
  {<5><6><7><8>cmtex8
   <9>cmtex9
   <10><10.95><12><14.4><17.28><20.74><24.88>cmtex10}{}
\DeclareFontShape{OT1}{cmtex}{m}{it}
  {<-> ssub * cmtt/m/it}{}
\newcommand{\texfamily}{\fontfamily{cmtex}\selectfont}
\DeclareFontShape{OT1}{cmtt}{bx}{n}
  {<5><6><7><8>cmtt8
   <9>cmbtt9
   <10><10.95><12><14.4><17.28><20.74><24.88>cmbtt10}{}
\DeclareFontShape{OT1}{cmtex}{bx}{n}
  {<-> ssub * cmtt/bx/n}{}
\newcommand{\tex}[1]{\text{\texfamily#1}}	% NEU

\newcommand{\Sp}{\hskip.33334em\relax}


\newcommand{\Conid}[1]{\mathit{#1}}
\newcommand{\Varid}[1]{\mathit{#1}}
\newcommand{\anonymous}{\kern0.06em \vbox{\hrule\@width.5em}}
\newcommand{\plus}{\mathbin{+\!\!\!+}}
\newcommand{\bind}{\mathbin{>\!\!\!>\mkern-6.7mu=}}
\newcommand{\rbind}{\mathbin{=\mkern-6.7mu<\!\!\!<}}% suggested by Neil Mitchell
\newcommand{\sequ}{\mathbin{>\!\!\!>}}
\renewcommand{\leq}{\leqslant}
\renewcommand{\geq}{\geqslant}
\usepackage{polytable}

%mathindent has to be defined
\@ifundefined{mathindent}%
  {\newdimen\mathindent\mathindent\leftmargini}%
  {}%

\def\resethooks{%
  \global\let\SaveRestoreHook\empty
  \global\let\ColumnHook\empty}
\newcommand*{\savecolumns}[1][default]%
  {\g@addto@macro\SaveRestoreHook{\savecolumns[#1]}}
\newcommand*{\restorecolumns}[1][default]%
  {\g@addto@macro\SaveRestoreHook{\restorecolumns[#1]}}
\newcommand*{\aligncolumn}[2]%
  {\g@addto@macro\ColumnHook{\column{#1}{#2}}}

\resethooks

\newcommand{\onelinecommentchars}{\quad-{}- }
\newcommand{\commentbeginchars}{\enskip\{-}
\newcommand{\commentendchars}{-\}\enskip}

\newcommand{\visiblecomments}{%
  \let\onelinecomment=\onelinecommentchars
  \let\commentbegin=\commentbeginchars
  \let\commentend=\commentendchars}

\newcommand{\invisiblecomments}{%
  \let\onelinecomment=\empty
  \let\commentbegin=\empty
  \let\commentend=\empty}

\visiblecomments

\newlength{\blanklineskip}
\setlength{\blanklineskip}{0.66084ex}

\newcommand{\hsindent}[1]{\quad}% default is fixed indentation
\let\hspre\empty
\let\hspost\empty
\newcommand{\NB}{\textbf{NB}}
\newcommand{\Todo}[1]{$\langle$\textbf{To do:}~#1$\rangle$}

\EndFmtInput
\makeatother
%
%
%
%
%
%
% This package provides two environments suitable to take the place
% of hscode, called "plainhscode" and "arrayhscode". 
%
% The plain environment surrounds each code block by vertical space,
% and it uses \abovedisplayskip and \belowdisplayskip to get spacing
% similar to formulas. Note that if these dimensions are changed,
% the spacing around displayed math formulas changes as well.
% All code is indented using \leftskip.
%
% Changed 19.08.2004 to reflect changes in colorcode. Should work with
% CodeGroup.sty.
%
\ReadOnlyOnce{polycode.fmt}%
\makeatletter

\newcommand{\hsnewpar}[1]%
  {{\parskip=0pt\parindent=0pt\par\vskip #1\noindent}}

% can be used, for instance, to redefine the code size, by setting the
% command to \small or something alike
\newcommand{\hscodestyle}{}

% The command \sethscode can be used to switch the code formatting
% behaviour by mapping the hscode environment in the subst directive
% to a new LaTeX environment.

\newcommand{\sethscode}[1]%
  {\expandafter\let\expandafter\hscode\csname #1\endcsname
   \expandafter\let\expandafter\endhscode\csname end#1\endcsname}

% "compatibility" mode restores the non-polycode.fmt layout.

\newenvironment{compathscode}%
  {\par\noindent
   \advance\leftskip\mathindent
   \hscodestyle
   \let\\=\@normalcr
   \let\hspre\(\let\hspost\)%
   \pboxed}%
  {\endpboxed\)%
   \par\noindent
   \ignorespacesafterend}

\newcommand{\compaths}{\sethscode{compathscode}}

% "plain" mode is the proposed default.
% It should now work with \centering.
% This required some changes. The old version
% is still available for reference as oldplainhscode.

\newenvironment{plainhscode}%
  {\hsnewpar\abovedisplayskip
   \advance\leftskip\mathindent
   \hscodestyle
   \let\hspre\(\let\hspost\)%
   \pboxed}%
  {\endpboxed%
   \hsnewpar\belowdisplayskip
   \ignorespacesafterend}

\newenvironment{oldplainhscode}%
  {\hsnewpar\abovedisplayskip
   \advance\leftskip\mathindent
   \hscodestyle
   \let\\=\@normalcr
   \(\pboxed}%
  {\endpboxed\)%
   \hsnewpar\belowdisplayskip
   \ignorespacesafterend}

% Here, we make plainhscode the default environment.

\newcommand{\plainhs}{\sethscode{plainhscode}}
\newcommand{\oldplainhs}{\sethscode{oldplainhscode}}
\plainhs

% The arrayhscode is like plain, but makes use of polytable's
% parray environment which disallows page breaks in code blocks.

\newenvironment{arrayhscode}%
  {\hsnewpar\abovedisplayskip
   \advance\leftskip\mathindent
   \hscodestyle
   \let\\=\@normalcr
   \(\parray}%
  {\endparray\)%
   \hsnewpar\belowdisplayskip
   \ignorespacesafterend}

\newcommand{\arrayhs}{\sethscode{arrayhscode}}

% The mathhscode environment also makes use of polytable's parray 
% environment. It is supposed to be used only inside math mode 
% (I used it to typeset the type rules in my thesis).

\newenvironment{mathhscode}%
  {\parray}{\endparray}

\newcommand{\mathhs}{\sethscode{mathhscode}}

% texths is similar to mathhs, but works in text mode.

\newenvironment{texthscode}%
  {\(\parray}{\endparray\)}

\newcommand{\texths}{\sethscode{texthscode}}

% The framed environment places code in a framed box.

\def\codeframewidth{\arrayrulewidth}
\RequirePackage{calc}

\newenvironment{framedhscode}%
  {\parskip=\abovedisplayskip\par\noindent
   \hscodestyle
   \arrayrulewidth=\codeframewidth
   \tabular{@{}|p{\linewidth-2\arraycolsep-2\arrayrulewidth-2pt}|@{}}%
   \hline\framedhslinecorrect\\{-1.5ex}%
   \let\endoflinesave=\\
   \let\\=\@normalcr
   \(\pboxed}%
  {\endpboxed\)%
   \framedhslinecorrect\endoflinesave{.5ex}\hline
   \endtabular
   \parskip=\belowdisplayskip\par\noindent
   \ignorespacesafterend}

\newcommand{\framedhslinecorrect}[2]%
  {#1[#2]}

\newcommand{\framedhs}{\sethscode{framedhscode}}

% The inlinehscode environment is an experimental environment
% that can be used to typeset displayed code inline.

\newenvironment{inlinehscode}%
  {\(\def\column##1##2{}%
   \let\>\undefined\let\<\undefined\let\\\undefined
   \newcommand\>[1][]{}\newcommand\<[1][]{}\newcommand\\[1][]{}%
   \def\fromto##1##2##3{##3}%
   \def\nextline{}}{\) }%

\newcommand{\inlinehs}{\sethscode{inlinehscode}}

% The joincode environment is a separate environment that
% can be used to surround and thereby connect multiple code
% blocks.

\newenvironment{joincode}%
  {\let\orighscode=\hscode
   \let\origendhscode=\endhscode
   \def\endhscode{\def\hscode{\endgroup\def\@currenvir{hscode}\\}\begingroup}
   %\let\SaveRestoreHook=\empty
   %\let\ColumnHook=\empty
   %\let\resethooks=\empty
   \orighscode\def\hscode{\endgroup\def\@currenvir{hscode}}}%
  {\origendhscode
   \global\let\hscode=\orighscode
   \global\let\endhscode=\origendhscode}%

\makeatother
\EndFmtInput
%
%
\ReadOnlyOnce{agda.fmt}%


\RequirePackage[T1]{fontenc}
\RequirePackage[utf8x]{inputenc}
\RequirePackage{ucs}
\RequirePackage{amsfonts}

\providecommand\mathbbm{\mathbb}

% TODO: Define more of these ...
\DeclareUnicodeCharacter{737}{\textsuperscript{l}}
\DeclareUnicodeCharacter{8718}{\ensuremath{\blacksquare}}
\DeclareUnicodeCharacter{8759}{::}
\DeclareUnicodeCharacter{9669}{\ensuremath{\triangleleft}}
\DeclareUnicodeCharacter{8799}{\ensuremath{\stackrel{\scriptscriptstyle ?}{=}}}
\DeclareUnicodeCharacter{10214}{\ensuremath{\llbracket}}
\DeclareUnicodeCharacter{10215}{\ensuremath{\rrbracket}}

% TODO: This is in general not a good idea.
\providecommand\textepsilon{$\epsilon$}
\providecommand\textmu{$\mu$}


%Actually, varsyms should not occur in Agda output.

% TODO: Make this configurable. IMHO, italics doesn't work well
% for Agda code.

\renewcommand\Varid[1]{\mathord{\textsf{#1}}}
\let\Conid\Varid
\newcommand\Keyword[1]{\textsf{\textbf{#1}}}
\EndFmtInput


\usepackage{pgfpages}

\usepackage{mathptmx}
\usepackage{ dsfont }

\usepackage{tikz}
\usepackage{amsmath,amssymb}
\usepackage{intcalc}

\usepackage{alltt}
\usepackage{xcolor}

\usetikzlibrary{arrows,shapes,calc,positioning,fit,matrix}
\usenavigationsymbolstemplate{}
% \setbeameroption{show notes on second screen=left} 
\setbeamertemplate{note page}[plain]
\setbeamerfont{note page}{size=\Large}

\setbeamersize{text margin left=.3cm, text margin right=.3cm}
\setbeamercolor{alerted text}{fg=red}


\usetheme{Warsaw}
\setbeamertemplate{footline}{}

\title{An Agda proof of the correctness of Valiant's algorithm for CF parsing}
\author{Thomas B{\aa}{\aa}th Sj{\"o}blom}
\institute{Chalmers University of Technology}
\date{}

\definecolor{darkgreen}{rgb}{0,0.4,0}

\begin{document}
\begin{frame}               % 1
\titlepage

\end{frame} %%%%%%%%%%%%%%%%%%%%%%%%%%%%%%%%%%%%%%%%%%%%%%%%%%%%%%%%%%%%%%%%%%%%
\begin{frame}{Overview}     % 2
\begin{itemize}
\item Parsing
  \begin{itemize}
  \item What is Parsing?
  \item Grammars
  \item Chart parsing % tie in to algebra
  \item Relation to transitive closure
  \end{itemize}
\item Valiant's Algorithm
  \begin{itemize}
  \item Motivation, then and now
  \item The algorithm
  \end{itemize}
\item Agda
  \begin{itemize}
  \item Introduction to Agda
  \item Implementation of Valiant's algorithm
  \item Sketch of Agda proof of Valiant's algorithm
  \end{itemize}  
\end{itemize}
% introduction
% example

% chart parsing
% transitive closure
% specification of transitive closure
\end{frame}


%\begin{frame}{Parsing}{Introduction}  % 3
% adding structure
%Parsing is adding structural information to a sequence of tokens.
% applications in compiler design, bioinformatics and computational linguistics
%\pause
% example
%\begin{equation*}
%  5 + 3 \times 2 + 7 \times (1 + 9) + 6
%\end{equation*}
% relevant to evaluating 
% numbers are factors
% 

% is the string generated by the grammar?
% introduction
% example
% uses
% why algebra? unified names

% chart parsing
% transitive closure
% specification of transitive closure
%\end{frame}


\begin{frame}{Parsing}{Introduction} % 4
\begin{itemize}
\item A grammar is a tuple $(N, \Sigma, P, S)$, where
  \begin{itemize}
  \item $N$ is the set of nonterminals.
  \item $\Sigma$ is the set of terminals, with $\Sigma \cap N = \emptyset$.
  \item $P$ is the set of production rules, each of the form $(\Sigma \cup N)^*N(\Sigma \cup N)^* \to (\Sigma \cup N)^*$
  \item $S \in N$ is the start symbol.
  \end{itemize}
\pause
\item We only consider grammars where the production rules are of the form (Chomsky normal form)
  \begin{itemize}
  \item $A \to BC$, where $A$, $B$, $C \in N$.
  \item $A \to \alpha$, where $A \in N$, $\alpha \in \Sigma$.
  \end{itemize}
\end{itemize}
\pause
Given a string, we want to see whether it can be generated from the start symbol, that is, whether there is a sequence of 
\end{frame}

\begin{frame}{Parsing}{Parse charts}
  Given a string of length $n$, we can make a matrix of all intermediate results into a matrix:
  \begin{equation*}
    \trc{C} = \begin{pmatrix}
      0 & a_{1 2} & a_{1 3} & \hdots & a_{1 n} \\
      0 & 0      & a_{2 3}  & \ldots & a_{2 n} \\
      \vdots & \vdots &\ddots & \ldots & \vdots\\
      0      & 0      & 0      & \ldots &a_{n-1 n}\\
      0      & 0      & 0      & \ldots &0
    \end{pmatrix}
  \end{equation*}
  where $a_{i j}$ is a set containing $A \in N$ iff $A$ generates the substring starting at the $i$th symbol, and ending at the $j$th.
\end{frame}

\begin{frame}{Parsing}{Parsing as transitive closure}
  \begin{itemize}
  \item We can multiply the sets by $x \cdot y = \{A : B \in x, C \in y, A \to BC \in P \}$.
    \pause
  \item Defining matrix multiplication as $(X \cdot Y)_{i j} = \bigcup_k X_{i k} Y_{k j}$ means that $(X \cdot Y)_{i j}$ contains $A$ if and only if $A$ is the product of a parse of the string from $i$ to $k$ in $X$ and from $k$ to $j$ in $Y$.
    \pause
    \item Hence, if $C$ is the matrix formed by placing the input just above the diagonal, the parse chart is the transitive closure of the input:
      \begin{equation*}
        \trc{C} = \trc{C}\trc{C} + C
      \end{equation*}
      \note{HUH, this is not very rigorous}
      A symbol is either an input symbol, or a combination of two parses.
  \end{itemize}
\end{frame}

\begin{frame}{Valiant's Algorithm}{Motivation} % 5
\begin{itemize}
  \item Then
    \begin{itemize}
    \item A theoretical algorithm for proving that parsing is as fast as matrix multiplication.
    \end{itemize}
  \item Now
    \begin{itemize}
    \item A practical (?) algorithm for parsing in parallel.
    \end{itemize}
  \item Idea
    \begin{itemize}
    \item Split string along the middle, recursively pars. Put together.
    \end{itemize}
\end{itemize}
\end{frame} %%%%%%%%%%%%%%%%%%%%%%%%%%%%%%%%%%%%%%%%%%%%%%%%%%%%%%%%%%%%%%%%%%%


\begin{frame}{Valiant's Algorithm}{The Algorithm} % 6
To compute the transitive closure $\tc(C)$ of
\begin{equation*}
  C = 
  \begin{pmatrix}
    U_1 & R_1 & A   & B \\
        & L_1 & C   & D \\
        &     & U_2 & R_2 \\
        &     &     & L_2
  \end{pmatrix}
\end{equation*}
\pause
Recursively, compute the transitive closures
\begin{equation*}
  \begin{pmatrix}
    \trc{U_1} & \trc{R_1} \\
          & \trc{L_1}
  \end{pmatrix}
  = \tc
  \begin{pmatrix}
    U_1 & R_1 \\
        & L_1
  \end{pmatrix}, 
   \begin{pmatrix}
    \trc{U_2} & \trc{R_2} \\
          & \trc{L_2}
  \end{pmatrix}
   = \tc
  \begin{pmatrix}
    U_2 & R_2 \\
        & L_2
  \end{pmatrix}.
\end{equation*}
\end{frame} %%%%%%%%%%%%%%%%%%%%%%%%%%%%%%%%%%%%%%%%%%%%%%%%%%%%%%%%%%%%%%%%%%%

\begin{frame}{Valiant's Algorithm}{The Algorithm} % 6

%% PERHAPS MAKE COOL ``ANIMATION''
Define $\overlap(X)$ to be the rectangular part of $\tc(X)$ and compute
\begin{align*}
  \trc{C} &= \overlap
  \begin{pmatrix}
    \trc{L_1} & C \\
        & \trc{U_2}
  \end{pmatrix}
  \\
  \trc{A} &= \overlap
  \begin{pmatrix}
    \trc{U_1} & A + \trc{R_1} \trc{C} \\
        & \trc{U_2}
  \end{pmatrix}
  \\
  \trc{D} &= \overlap
  \begin{pmatrix}
    \trc{L_1} & D + \trc{C} \trc{R_2} \\
        & \trc{L_2}
  \end{pmatrix}
  \\
  \trc{B} &= \overlap
  \begin{pmatrix}
    \trc{U_1} & B + \trc{R_1} \trc{D} + \trc{A} \trc{R_2} \\
        & \trc{L_2}
  \end{pmatrix}
\end{align*}
\pause
\end{frame} %%%%%%%%%%%%%%%%%%%%%%%%%%%%%%%%%%%%%%%%%%%%%%%%%%%%%%%%%%%%%%%%%%%



\begin{frame}{Valiant's Algorithm}{The Algorithm}  % 7
Putting things together, the transitive closure of $C$ is 
\begin{equation*}
  C = 
  \begin{pmatrix}
    \trc{U_1} & \trc{R_1} & \trc{A}   & \trc{B} \\
        & \trc{L_1} & \trc{C}   & \trc{D} \\
        &     & \trc{U_2} & \trc{R_2} \\
        &     &     & \trc{L_2}
  \end{pmatrix}
\end{equation*}
\end{frame} %%%%%%%%%%%%%%%%%%%%%%%%%%%%%%%%%%%%%%%%%%%%%%%%%%%%%%%%%%%%%%%%%%%



\begin{frame}{Agda} % 8
Agda is a dependently typed functional language.
\pause
\begin{hscode}\SaveRestoreHook
\column{B}{@{}>{\hspre}l<{\hspost}@{}}%
\column{3}{@{}>{\hspre}l<{\hspost}@{}}%
\column{9}{@{}>{\hspre}l<{\hspost}@{}}%
\column{E}{@{}>{\hspre}l<{\hspost}@{}}%
\>[B]{}\Keyword{data}\;\Conid{ℕ}\;\mathbin{:}\;\Conid{Set}\;\Keyword{where}{}\<[E]%
\\
\>[B]{}\hsindent{3}{}\<[3]%
\>[3]{}\Varid{zero}\;{}\<[9]%
\>[9]{}\mathbin{:}\;\Conid{ℕ}{}\<[E]%
\\
\>[B]{}\hsindent{3}{}\<[3]%
\>[3]{}\Varid{suc}\;{}\<[9]%
\>[9]{}\mathbin{:}\;\Conid{ℕ}\;\Varid{→}\;\Conid{ℕ}{}\<[E]%
\ColumnHook
\end{hscode}\resethooks
\note{Syntax--unicode}
\pause
\vspace{-1cm}
\begin{hscode}\SaveRestoreHook
\column{B}{@{}>{\hspre}l<{\hspost}@{}}%
\column{8}{@{}>{\hspre}l<{\hspost}@{}}%
\column{E}{@{}>{\hspre}l<{\hspost}@{}}%
\>[B]{}\Varid{\char95 +\char95 }\;\mathbin{:}\;\Conid{ℕ}\;\Varid{→}\;\Conid{ℕ}\;\Varid{→}\;\Conid{ℕ}{}\<[E]%
\\
\>[B]{}\Varid{zero}\;{}\<[8]%
\>[8]{}\Varid{+}\;\Varid{n}\;\mathrel{=}\;\Varid{n}{}\<[E]%
\\
\>[B]{}\Varid{suc}\;\Varid{m}\;{}\<[8]%
\>[8]{}\Varid{+}\;\Varid{n}\;\mathrel{=}\;\Varid{suc}\;(\Varid{m}\;\Varid{+}\;\Varid{n}){}\<[E]%
\ColumnHook
\end{hscode}\resethooks
\note{recursion, underscores}
%Agda is total. All Agda programs terminate.
\end{frame} %%%%%%%%%%%%%%%%%%%%%%%%%%%%%%%%%%%%%%%%%%%%%%%%%%%%%%%%%%%%%%%%%%%
% intro:
% simple def of say nat
% function definition on nat

% define our matrices and triangles
% 
\begin{frame}{Agda}{Implementing Valiant's Algorithm} % 9
Datatype for matrices:
\begin{hscode}\SaveRestoreHook
\column{B}{@{}>{\hspre}l<{\hspost}@{}}%
\column{3}{@{}>{\hspre}l<{\hspost}@{}}%
\column{8}{@{}>{\hspre}l<{\hspost}@{}}%
\column{E}{@{}>{\hspre}l<{\hspost}@{}}%
\>[B]{}\Keyword{data}\;\Conid{Split}\;\mathbin{:}\;\Conid{Set}\;\Keyword{where}{}\<[E]%
\\
\>[B]{}\hsindent{3}{}\<[3]%
\>[3]{}\Varid{one}\;{}\<[8]%
\>[8]{}\mathbin{:}\;\Conid{Split}{}\<[E]%
\\
\>[B]{}\hsindent{3}{}\<[3]%
\>[3]{}\Varid{bin}\;{}\<[8]%
\>[8]{}\mathbin{:}\;\Conid{Split}\;\Varid{→}\;\Conid{Split}\;\Varid{→}\;\Conid{Split}{}\<[E]%
\ColumnHook
\end{hscode}\resethooks
\begin{hscode}\SaveRestoreHook
\column{B}{@{}>{\hspre}l<{\hspost}@{}}%
\column{3}{@{}>{\hspre}l<{\hspost}@{}}%
\column{9}{@{}>{\hspre}l<{\hspost}@{}}%
\column{28}{@{}>{\hspre}l<{\hspost}@{}}%
\column{E}{@{}>{\hspre}l<{\hspost}@{}}%
\>[B]{}\Keyword{data}\;\Conid{Mat}\;\mathbin{:}\;\Conid{Split}\;\Varid{→}\;\Conid{Split}\;\Varid{→}\;\Conid{Set}\;\Keyword{where}{}\<[E]%
\\
\>[B]{}\hsindent{3}{}\<[3]%
\>[3]{}\Varid{sing}\;{}\<[9]%
\>[9]{}\mathbin{:}\;\Conid{R}\;\Varid{→}\;\Conid{Mat}\;\Varid{one}\;\Varid{one}{}\<[E]%
\\
\>[B]{}\hsindent{3}{}\<[3]%
\>[3]{}\Varid{quad}\;{}\<[9]%
\>[9]{}\mathbin{:}\;\Varid{∀}\;\{\mskip1.5mu \Varid{r₁}\;\Varid{r₂}\;\Varid{c₁}\;\Varid{c₂}\mskip1.5mu\}\;{}\<[28]%
\>[28]{}\Varid{→}\;\Conid{Mat}\;\Varid{r₁}\;\Varid{c₁}\;\Varid{→}\;\Conid{Mat}\;\Varid{r₁}\;\Varid{c₂}\;{}\<[E]%
\\
\>[28]{}\Varid{→}\;\Conid{Mat}\;\Varid{r₂}\;\Varid{c₁}\;\Varid{→}\;\Conid{Mat}\;\Varid{r₂}\;\Varid{c₂}\;{}\<[E]%
\\
\>[28]{}\Varid{→}\;\Conid{Mat}\;(\Varid{bin}\;\Varid{r₁}\;\Varid{r₂})\;(\Varid{bin}\;\Varid{c₁}\;\Varid{c₂}){}\<[E]%
\ColumnHook
\end{hscode}\resethooks
\end{frame} %%%%%%%%%%%%%%%%%%%%%%%%%%%%%%%%%%%%%%%%%%%%%%%%%%%%%%%%%%%%%%%%%%%

\begin{frame}{Agda}{Implementing Valiant's Algorithm} % 10
Datatype for triangles:
\begin{hscode}\SaveRestoreHook
\column{B}{@{}>{\hspre}l<{\hspost}@{}}%
\column{3}{@{}>{\hspre}l<{\hspost}@{}}%
\column{8}{@{}>{\hspre}l<{\hspost}@{}}%
\column{27}{@{}>{\hspre}l<{\hspost}@{}}%
\column{E}{@{}>{\hspre}l<{\hspost}@{}}%
\>[B]{}\Keyword{data}\;\Conid{Tri}\;\mathbin{:}\;\Conid{Split}\;\Varid{→}\;\Conid{Set}\;\Keyword{where}{}\<[E]%
\\
\>[B]{}\hsindent{3}{}\<[3]%
\>[3]{}\Varid{one}\;{}\<[8]%
\>[8]{}\mathbin{:}\;\Conid{Tri}\;\Varid{one}{}\<[E]%
\\
\>[B]{}\hsindent{3}{}\<[3]%
\>[3]{}\Varid{tri}\;{}\<[8]%
\>[8]{}\mathbin{:}\;\Varid{∀}\;\{\mskip1.5mu \Varid{a}\;\Varid{b}\mskip1.5mu\}\;\Varid{→}\;\Conid{Tri}\;\Varid{a}\;{}\<[27]%
\>[27]{}\Varid{→}\;\Conid{Mat}\;\Varid{a}\;\Varid{b}\;{}\<[E]%
\\
\>[27]{}\Varid{→}\;\Conid{Tri}\;\Varid{b}\;{}\<[E]%
\\
\>[27]{}\Varid{→}\;\Conid{Tri}\;(\Varid{bin}\;\Varid{a}\;\Varid{b}){}\<[E]%
\ColumnHook
\end{hscode}\resethooks
\end{frame} %%%%%%%%%%%%%%%%%%%%%%%%%%%%%%%%%%%%%%%%%%%%%%%%%%%%%%%%%%%%%%%%%%%


\begin{frame}{Agda}{Implementing Valiant's Algorithm} % 11
Operations:
\vspace{-0.5cm}
\begin{hscode}\SaveRestoreHook
\column{B}{@{}>{\hspre}l<{\hspost}@{}}%
\column{8}{@{}>{\hspre}l<{\hspost}@{}}%
\column{15}{@{}>{\hspre}l<{\hspost}@{}}%
\column{35}{@{}>{\hspre}l<{\hspost}@{}}%
\column{E}{@{}>{\hspre}l<{\hspost}@{}}%
\>[B]{}\Varid{\char95 +\char95 }\;\mathbin{:}\;\Varid{∀}\;\{\mskip1.5mu \Varid{a}\;\Varid{b}\mskip1.5mu\}\;\Varid{→}\;\Conid{Mat}\;\Varid{a}\;\Varid{b}\;\Varid{→}\;\Conid{Mat}\;\Varid{a}\;\Varid{b}\;\Varid{→}\;\Conid{Mat}\;\Varid{a}\;\Varid{b}{}\<[E]%
\\
\>[B]{}\Varid{sing}\;\Varid{x}\;{}\<[15]%
\>[15]{}\Varid{+}\;\Varid{sing}\;\Varid{x'}\;{}\<[35]%
\>[35]{}\mathrel{=}\;\Varid{sing}\;(\Varid{x}\;\Conid{R+}\;\Varid{x'}){}\<[E]%
\\
\>[B]{}\Varid{quad}\;\Conid{A}\;\Conid{B}\;\Conid{C}\;\Conid{D}\;{}\<[15]%
\>[15]{}\Varid{+}\;\Varid{quad}\;\Conid{A'}\;\Conid{B'}\;\Conid{C'}\;\Conid{D'}\;{}\<[35]%
\>[35]{}\mathrel{=}\;\Varid{quad}\;{}\<[E]%
\\
\>[B]{}\hsindent{8}{}\<[8]%
\>[8]{}(\Conid{A}\;\Varid{+}\;\Conid{A'})\;(\Conid{B}\;\Varid{+}\;\Conid{B'})\;{}\<[E]%
\\
\>[B]{}\hsindent{8}{}\<[8]%
\>[8]{}(\Conid{C}\;\Varid{+}\;\Conid{C'})\;(\Conid{D}\;\Varid{+}\;\Conid{D'}){}\<[E]%
\ColumnHook
\end{hscode}\resethooks

\begin{hscode}\SaveRestoreHook
\column{B}{@{}>{\hspre}l<{\hspost}@{}}%
\column{8}{@{}>{\hspre}l<{\hspost}@{}}%
\column{15}{@{}>{\hspre}l<{\hspost}@{}}%
\column{35}{@{}>{\hspre}l<{\hspost}@{}}%
\column{E}{@{}>{\hspre}l<{\hspost}@{}}%
\>[B]{}\Varid{\char95 *\char95 }\;\mathbin{:}\;\Varid{∀}\;\{\mskip1.5mu \Varid{a}\;\Varid{b}\;\Varid{c}\mskip1.5mu\}\;\Varid{→}\;\Conid{Mat}\;\Varid{a}\;\Varid{b}\;\Varid{→}\;\Conid{Mat}\;\Varid{b}\;\Varid{c}\;\Varid{→}\;\Conid{Mat}\;\Varid{a}\;\Varid{c}{}\<[E]%
\\
\>[B]{}\Varid{sing}\;\Varid{x}\;{}\<[15]%
\>[15]{}\Varid{*}\;\Varid{sing}\;\Varid{x'}\;{}\<[35]%
\>[35]{}\mathrel{=}\;\Varid{sing}\;(\Varid{x}\;\Conid{R*}\;\Varid{x'}){}\<[E]%
\\
\>[B]{}\Varid{quad}\;\Conid{A}\;\Conid{B}\;\Conid{C}\;\Conid{D}\;{}\<[15]%
\>[15]{}\Varid{*}\;\Varid{quad}\;\Conid{A'}\;\Conid{B'}\;\Conid{C'}\;\Conid{D'}\;{}\<[35]%
\>[35]{}\mathrel{=}\;\Varid{quad}\;{}\<[E]%
\\
\>[B]{}\hsindent{8}{}\<[8]%
\>[8]{}(\Conid{A}\;\Varid{*}\;\Conid{A'}\;\Varid{+}\;\Conid{B}\;\Varid{*}\;\Conid{C'})\;(\Conid{A}\;\Varid{*}\;\Conid{B'}\;\Varid{+}\;\Conid{B}\;\Varid{*}\;\Conid{D'})\;{}\<[E]%
\\
\>[B]{}\hsindent{8}{}\<[8]%
\>[8]{}(\Conid{C}\;\Varid{*}\;\Conid{A'}\;\Varid{+}\;\Conid{D}\;\Varid{*}\;\Conid{C'})\;(\Conid{C}\;\Varid{*}\;\Conid{B'}\;\Varid{+}\;\Conid{D}\;\Varid{*}\;\Conid{D'}){}\<[E]%
\ColumnHook
\end{hscode}\resethooks
\end{frame} %%%%%%%%%%%%%%%%%%%%%%%%%%%%%%%%%%%%%%%%%%%%%%%%%%%%%%%%%%%%%%%%%%%


\begin{frame}{Agda}{Implementing Valiant's Algorithm} % 11
\begin{hscode}\SaveRestoreHook
\column{B}{@{}>{\hspre}l<{\hspost}@{}}%
\column{3}{@{}>{\hspre}l<{\hspost}@{}}%
\column{10}{@{}>{\hspre}l<{\hspost}@{}}%
\column{14}{@{}>{\hspre}l<{\hspost}@{}}%
\column{27}{@{}>{\hspre}l<{\hspost}@{}}%
\column{42}{@{}>{\hspre}l<{\hspost}@{}}%
\column{60}{@{}>{\hspre}l<{\hspost}@{}}%
\column{E}{@{}>{\hspre}l<{\hspost}@{}}%
\>[B]{}\Varid{overlap}\;\mathbin{:}\;\Varid{∀}\;\{\mskip1.5mu \Varid{a}\;\Varid{b}\mskip1.5mu\}\;\Varid{→}\;\Conid{Tri}\;\Varid{a}\;\Varid{→}\;\Conid{Mat}\;\Varid{a}\;\Varid{b}\;\Varid{→}\;\Conid{Tri}\;\Varid{b}\;\Varid{→}\;\Conid{Mat}\;\Varid{a}\;\Varid{b}{}\<[E]%
\\
\>[B]{}\Varid{overlap}\;\Varid{one}\;{}\<[27]%
\>[27]{}(\Varid{sing}\;\Varid{x})\;{}\<[42]%
\>[42]{}\Varid{one}\;{}\<[60]%
\>[60]{}\mathrel{=}\;\Varid{sing}\;\Varid{x}{}\<[E]%
\\
\>[B]{}\Varid{overlap}\;(\Varid{tri}\;\Conid{U₁ˣ}\;\Conid{R₁ˣ}\;\Conid{L₁ˣ})\;(\Varid{quad}\;\Conid{A}\;\Conid{B}\;\Conid{C}\;\Conid{D})\;(\Varid{tri}\;\Conid{U₂ˣ}\;\Conid{R₂ˣ}\;\Conid{L₂ˣ})\;\mathrel{=}\;\Varid{quad}\;\Conid{Aˣ}\;\Conid{Bˣ}\;\Conid{Cˣ}\;\Conid{Dˣ}{}\<[E]%
\\
\>[B]{}\hsindent{3}{}\<[3]%
\>[3]{}\Keyword{where}\;{}\<[10]%
\>[10]{}\Conid{Cˣ}\;{}\<[14]%
\>[14]{}\mathrel{=}\;\Varid{overlap}\;\Conid{L₁ˣ}\;\Conid{C}\;\Conid{U₂ˣ}{}\<[E]%
\\
\>[10]{}\Conid{Aˣ}\;{}\<[14]%
\>[14]{}\mathrel{=}\;\Varid{overlap}\;\Conid{U₁ˣ}\;(\Conid{A}\;\Varid{+}\;\Conid{R₁ˣ}\;\Varid{*}\;\Conid{Cˣ})\;\Conid{U₂ˣ}{}\<[E]%
\\
\>[10]{}\Conid{Dˣ}\;{}\<[14]%
\>[14]{}\mathrel{=}\;\Varid{overlap}\;\Conid{L₁ˣ}\;(\Conid{D}\;\Varid{+}\;\Conid{Cˣ}\;\Varid{*}\;\Conid{R₂ˣ})\;\Conid{L₂ˣ}{}\<[E]%
\\
\>[10]{}\Conid{Bˣ}\;{}\<[14]%
\>[14]{}\mathrel{=}\;\Varid{overlap}\;\Conid{U₁ˣ}\;(\Conid{B}\;\Varid{+}\;\Conid{R₁ˣ}\;\Varid{*}\;\Conid{Dˣ}\;\Varid{+}\;\Conid{Aˣ}\;\Varid{*}\;\Conid{R₂ˣ})\;\Conid{L₂ˣ}{}\<[E]%
\ColumnHook
\end{hscode}\resethooks
\end{frame} %%%%%%%%%%%%%%%%%%%%%%%%%%%%%%%%%%%%%%%%%%%%%%%%%%%%%%%%%%%%%%%%%%%

\begin{frame}{Agda}{Implementing Valiant's Algorithm} % 11
\begin{hscode}\SaveRestoreHook
\column{B}{@{}>{\hspre}l<{\hspost}@{}}%
\column{3}{@{}>{\hspre}l<{\hspost}@{}}%
\column{10}{@{}>{\hspre}l<{\hspost}@{}}%
\column{14}{@{}>{\hspre}l<{\hspost}@{}}%
\column{16}{@{}>{\hspre}l<{\hspost}@{}}%
\column{E}{@{}>{\hspre}l<{\hspost}@{}}%
\>[B]{}\Varid{v}\;\mathbin{:}\;\Varid{∀}\;\{\mskip1.5mu \Varid{s}\mskip1.5mu\}\;\Varid{→}\;\Conid{Tri}\;\Varid{s}\;\Varid{→}\;\Conid{Tri}\;\Varid{s}{}\<[E]%
\\
\>[B]{}\Varid{v}\;\Varid{one}\;{}\<[16]%
\>[16]{}\mathrel{=}\;\Varid{one}{}\<[E]%
\\
\>[B]{}\Varid{v}\;(\Varid{tri}\;\Conid{U}\;\Conid{R}\;\Conid{L})\;{}\<[16]%
\>[16]{}\mathrel{=}\;\Varid{tri}\;\Conid{U⁺}\;(\Varid{overlap}\;\Conid{U⁺}\;\Conid{R}\;\Conid{L⁺})\;\Conid{L⁺}{}\<[E]%
\\
\>[B]{}\hsindent{3}{}\<[3]%
\>[3]{}\Keyword{where}\;{}\<[10]%
\>[10]{}\Conid{U⁺}\;{}\<[14]%
\>[14]{}\mathrel{=}\;\Varid{v}\;\Conid{U}{}\<[E]%
\\
\>[10]{}\Conid{L⁺}\;{}\<[14]%
\>[14]{}\mathrel{=}\;\Varid{v}\;\Conid{L}{}\<[E]%
\ColumnHook
\end{hscode}\resethooks
\end{frame} %%%%%%%%%%%%%%%%%%%%%%%%%%%%%%%%%%%%%%%%%%%%%%%%%%%%%%%%%%%%%%%%%%%


\begin{frame}{Agda}{Proving things in Agda} % 12
The Curry--Howard correspondence:

A propositions can be seen as the type containing all proofs of the proposition.
\pause
\begin{hscode}\SaveRestoreHook
\column{B}{@{}>{\hspre}l<{\hspost}@{}}%
\column{3}{@{}>{\hspre}l<{\hspost}@{}}%
\column{8}{@{}>{\hspre}l<{\hspost}@{}}%
\column{27}{@{}>{\hspre}l<{\hspost}@{}}%
\column{E}{@{}>{\hspre}l<{\hspost}@{}}%
\>[B]{}\Keyword{data}\;\Varid{\char95 ≤\char95 }\;\mathbin{:}\;\Conid{ℕ}\;\Varid{→}\;\Conid{ℕ}\;\Varid{→}\;\Conid{Set}\;\Keyword{where}{}\<[E]%
\\
\>[B]{}\hsindent{3}{}\<[3]%
\>[3]{}\Varid{z≤n}\;{}\<[8]%
\>[8]{}\mathbin{:}\;\Varid{∀}\;\{\mskip1.5mu \Varid{n}\mskip1.5mu\}\;{}\<[27]%
\>[27]{}\Varid{→}\;\Varid{zero}\;\Varid{≤}\;\Varid{n}{}\<[E]%
\\
\>[B]{}\hsindent{3}{}\<[3]%
\>[3]{}\Varid{s≤s}\;{}\<[8]%
\>[8]{}\mathbin{:}\;\Varid{∀}\;\{\mskip1.5mu \Varid{m}\;\Varid{n}\mskip1.5mu\}\;\Varid{→}\;\Varid{m}\;\Varid{≤}\;\Varid{n}\;{}\<[27]%
\>[27]{}\Varid{→}\;\Varid{suc}\;\Varid{m}\;\Varid{≤}\;\Varid{suc}\;\Varid{n}{}\<[E]%
\ColumnHook
\end{hscode}\resethooks
\note{what is indexed? type?, two different canonical proofs}
\pause
To prove something a proposition is to give an inhabitant of the type.
\end{frame} %%%%%%%%%%%%%%%%%%%%%%%%%%%%%%%%%%%%%%%%%%%%%%%%%%%%%%%%%%%%%%%%%%%

\begin{frame}{Agda}{Specification of Valiant's algorithm} % 13
\begin{itemize}
\item A relation from triangles to triangles:
\begin{hscode}\SaveRestoreHook
\column{B}{@{}>{\hspre}l<{\hspost}@{}}%
\column{E}{@{}>{\hspre}l<{\hspost}@{}}%
\>[B]{}\Varid{\char95 is-tc-of\char95 }\;\mathbin{:}\;\Varid{∀}\;\{\mskip1.5mu \Varid{s}\mskip1.5mu\}\;\Varid{→}\;\Conid{Tri}\;\Varid{s}\;\Varid{→}\;\Conid{Tri}\;\Varid{s}\;\Varid{→}\;\Conid{Set}{}\<[E]%
\\
\>[B]{}\Conid{Cˣ}\;\Varid{is-tc-of}\;\Conid{C}\;\mathrel{=}\;\Conid{Cˣ}\;\Varid{≈}\;\Conid{Cˣ}\;\Varid{*}\;\Conid{Cˣ}\;\Varid{+}\;\Conid{C}{}\<[E]%
\ColumnHook
\end{hscode}\resethooks
\pause
\item To prove the correctness, find an element of type
\begin{hscode}\SaveRestoreHook
\column{B}{@{}>{\hspre}l<{\hspost}@{}}%
\column{E}{@{}>{\hspre}l<{\hspost}@{}}%
\>[B]{}\Varid{∀}\;\{\mskip1.5mu \Varid{s}\mskip1.5mu\}\;\{\mskip1.5mu \Conid{C}\;\mathbin{:}\;\Conid{Tri}\;\Varid{s}\mskip1.5mu\}\;\Varid{→}\;\Varid{v}\;\Conid{C}\;\Varid{is-tc-of}\;\Conid{C}{}\<[E]%
\ColumnHook
\end{hscode}\resethooks
\end{itemize}
\end{frame} %%%%%%%%%%%%%%%%%%%%%%%%%%%%%%%%%%%%%%%%%%%%%%%%%%%%%%%%%%%%%%%%%%%

\begin{frame}{Agda}{Sketch of proof} % 14
\begin{itemize}
  \item We pattern match on the \ensuremath{\Conid{Tri}}.
    \begin{itemize}
    \item \ensuremath{\Varid{one}}, should have type
      \begin{hscode}\SaveRestoreHook
\column{B}{@{}>{\hspre}l<{\hspost}@{}}%
\column{9}{@{}>{\hspre}l<{\hspost}@{}}%
\column{E}{@{}>{\hspre}l<{\hspost}@{}}%
\>[9]{}\Varid{one}\;\Varid{≈}\;\Varid{one}\;\Varid{*}\;\Varid{one}\;\Varid{+}\;\Varid{one}{}\<[E]%
\ColumnHook
\end{hscode}\resethooks
    \item \ensuremath{\Varid{tri}}, we get that the overlap function should satisfy
      \begin{hscode}\SaveRestoreHook
\column{B}{@{}>{\hspre}l<{\hspost}@{}}%
\column{9}{@{}>{\hspre}l<{\hspost}@{}}%
\column{24}{@{}>{\hspre}c<{\hspost}@{}}%
\column{24E}{@{}l@{}}%
\column{E}{@{}>{\hspre}l<{\hspost}@{}}%
\>[9]{}\Conid{Rˣ}\;\Varid{≈}\;\Conid{U}\;\Varid{*}\;\Conid{Rˣ}\;\Varid{+}\;{}\<[24]%
\>[24]{}\Conid{R},{}\<[24E]%
\ColumnHook
\end{hscode}\resethooks
      which when we expand it gives us exactly the definition of it.
    \end{itemize}
\end{itemize}
\end{frame} %%%%%%%%%%%%%%%%%%%%%%%%%%%%%%%%%%%%%%%%%%%%%%%%%%%%%%%%%%%%%%%%%%%

\end{document}
